\documentclass[12pt]{exam}
\usepackage[utf8]{inputenc}
\usepackage{amsmath,amstext,amsthm,amssymb,amsxtra, graphicx}
\usepackage[top=1.5in, bottom=1.5in, left=1.25in, right=1.25in]	{geometry}
%\usepackage[normalem]{ulem}
\usepackage{txfonts} % pxfonts txfonts 
\usepackage[T1]{fontenc}
\usepackage{lmodern}
\renewcommand*\familydefault{\sfdefault}
 \usepackage{euler}   % better than the option below
\usepackage{pdfsync}
\usepackage{multicol}
\newcommand{\ci}[1]{_{ {}_{\scriptstyle #1}}}
\graphicspath{ {images/} }


\newcommand{\norm}[1]{\ensuremath{\left\|#1\right\|}}
\newcommand{\abs}[1]{\ensuremath{\left\vert#1\right\vert}}
\newcommand{\ip}[2]{\ensuremath{\left\langle#1,#2\right\rangle}}
\newcommand{\p}{\ensuremath{\partial}}
\newcommand{\pr}{\mathcal{P}}

\newcommand{\pbar}{\ensuremath{\bar{\partial}}}
\newcommand{\db}{\overline\partial}
\newcommand{\D}{\mathbb{D}}
\newcommand{\B}{\mathbb{B}}
\newcommand{\Sp}{\mathbb{S}}
\newcommand{\T}{\mathbb{T}}
\newcommand{\R}{\mathbb{R}}
\newcommand{\Z}{\mathbb{Z}}
\newcommand{\C}{\mathbb{C}}
\newcommand{\N}{\mathbb{N}}
\newcommand{\Q}{\mathbb{Q}}
\newcommand{\mQ}{\mathcal{Q}}
\newcommand{\mS}{\mathcal{S}}
\newcommand{\scrH}{\mathcal{H}}
\newcommand{\scrL}{\mathcal{L}}
\newcommand{\td}{\widetilde\Delta}
\newcommand{\pw}{\text{PW}}
\newcommand{\esup}{\text{ess.sup}}
\newcommand{\Tn}{\mathcal{T}_n}
\newcommand{\Bn}{\mathbb{B}_n}
\newcommand{\rt}{\mathcal{O}}
\newcommand{\avg}[1]{\langle #1 \rangle}
\newcommand{\one}{\mathbbm{1}}
\newcommand{\eps}{\varepsilon}
\newcommand{\grad}{\nabla}

\newcommand{\La}{\langle }
\newcommand{\Ra}{\rangle }
\newcommand{\rk}{\operatorname{rk}}
\newcommand{\card}{\operatorname{card}}
\newcommand{\ran}{\operatorname{Ran}}
\newcommand{\osc}{\operatorname{OSC}}
\newcommand{\im}{\operatorname{Im}}
\newcommand{\re}{\operatorname{Re}}
\newcommand{\tr}{\operatorname{tr}}
\newcommand{\vf}{\varphi}
\newcommand{\f}[2]{\ensuremath{\frac{#1}{#2}}}

\newcommand{\kzp}{k_z^{(p,\alpha)}}
\newcommand{\klp}{k_{\lambda_i}^{(p,\alpha)}}
\newcommand{\TTp}{\mathcal{T}_p}
\newcommand{\m}[1]{\mathcal{#1}}
\newcommand{\md}{\mathcal{D}}
\newcommand{\qan}{\abs{Q}^{\alpha/n}}
\newcommand{\sbump}[2]{[[ #1,#2 ]]}
\newcommand{\mbump}[2]{\lceil #1,#2 \rceil}
\newcommand{\cbump}[2]{\lfloor #1,#2 \rfloor}

\newcommand{\hn}{{2}}
\newcommand{\dd}{{09-21}}
\newcommand{\class}{Aero 417}
\newcommand{\term}{Fall 2024}
\newcommand{\examnum}{Homework \hn: Due \dd}
\newcommand{\examdate}{}
\newcommand{\timelimit}{75 Minutes}
\newcommand{\vc}[3]{\langle #1,#2,#3\rangle}
\newcommand*{\vv}[1]{\vec{\mkern0mu#1}}
\newcommand{\bv}[1]{\boldsymbol{#1}}
\newcommand{\hide}[1]{}
\newcommand{\uvec}[1]{\boldsymbol{\hat{\textbf{#1}}}}
\newcommand{\vex}[1]{\boldsymbol{{\textbf{#1}}}}
\newcommand{\px}{\frac{\partial}{\partial x}}
\newcommand{\py}{\frac{\partial}{\partial y}}
\newcommand{\pt}{\frac{\partial}{\partial t}}
\newcommand{\pxx}{\frac{\partial^2}{\partial x^2}}
\newcommand{\pyy}{\frac{\partial^2}{\partial y^2}}
\newcommand{\ptt}{\frac{\partial^2}{\partial t^2}}


\pagestyle{head}
\firstpageheader{}{}{}
\runningheader{\class}{ Page \thepage\ of \numpages}{\examnum}
\runningheadrule

\makeatletter
\renewcommand*\env@matrix[1][*\c@MaxMatrixCols c]{%
  \hskip -\arraycolsep
  \let\@ifnextchar\new@ifnextchar
  \array{#1}}
\makeatother

\printanswers
\begin{document}

\noindent
\begin{tabular*}{\textwidth}{l @{\extracolsep{\fill}} r @{\extracolsep{6pt}} l}
\textbf{\class} & \textbf{Name:} & \makebox[2in]{\bf{Benjamin Tollison}}\\
\end{tabular*}\\
\rule[2ex]{\textwidth}{2pt}
%
\begin{questions}
\begin{question}
A jet engine is traveling through the air with the forward velocity of 300 m/s. 
The exhaust gases leave the nozzle with an exit velocity of 800 m/s with respect to the nozzle. If the mass flow rate through the engine is 10 kg/s, determine the jet engine thrust.
The exit plane static pressure is 80 kPa, inlet plane static pressure is 20 kPa, ambient pressure surrounding the engine is 20 kPa, and the exit plane area is 4.0 m$^2$.  
\end{question}
\begin{solutionorbox}[\stretch{1}]
\begin{align*}
\begin{cases}
  V_\infty = 300  \frac{m}{s}\\
  V_e = 800  \frac{m}{s}\\
  P_e = 80  \text{kPa}\\
  P_{\text{atm}} = 20  \text{kPa} \\
  A_e = 4 \, m^2 \\
  \dot{m} = 10  \frac{kg}{s}
\end{cases}
\end{align*}

\[T = \dot{m} V_e - \dot{m} V_\infty + A_e (P_e - P_{\text{atm}})\]
\[\therefore T = 245 \text{kN}\]

\end{solutionorbox}

\newpage 
\begin{question}
Describe the differences between Brayton Cycle and a Real Gas Turbine Cycle. Make
diagrams to explain the losses associated with a real engine.
\end{question}
\begin{solutionorbox}[\stretch{1}]
\[\]
\includegraphics[width=\linewidth]{brayton v real cycle.png}
The first losses occur in the compressor from stage 1->2 due to the 
flow not being reversible. The combustion process is not completely isobaric.
The process in the turbine is also not reversible.
\end{solutionorbox}


\newpage 
\begin{question}
Draw the T-s diagram and determine the turbine shaft power, and the air-fuel ratio
\end{question}
\begin{solutionorbox}[\stretch{1}]
With all the given infomation the only equations used to find the shaft power were:
\[P_{02} = rP_{01}\]
\[T_{02} = T_{01}\left[1 + \frac{1}{\eta_c}\left(r^\frac{\gamma-1}{\gamma} - 1\right)\right]\]
\[W_* = \frac{1}{\eta_*}cp_*(T_i - T_j)\]
\[\text{P} = \dot{m}_{gas}(W_T - W_C)\]
\(\therefore \text{P}_{shaft} = 5740.3282\) [kW] 

The entropy starting values were pulled from Dr.Cizmas' textbook from the 
air tables for \(s_1\) and stoichiometric tables for \(s_3\). The following equation
was used to find the other two values:
\[s_i - s_j = cp_*\ln{\frac{T_i}{T_j}} - R\ln{\frac{P_i}{P_j}}\]
which then produces the following T-s graph.
\begin{center}
\includegraphics[width=350pt]{t-s-graph.png}
\end{center}
The ideal fuel to air ratio was pulled from the 5th set of lecture slides:
\[\Delta{t_c} = 588.4864 \rightarrow f \approx 0.014\]
\[\therefore f^{-1} \approx 71.429\]

\end{solutionorbox}

\end{questions}


\end{document}
