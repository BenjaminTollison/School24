\documentclass[12pt]{exam}
\usepackage[utf8]{inputenc}
\usepackage{cancel}
\usepackage{amsmath,amstext,amsthm,amssymb,amsxtra, graphicx}
\usepackage[top=1.5in, bottom=1.5in, left=1.25in, right=1.25in]	{geometry}
%\usepackage[normalem]{ulem}
\usepackage{pdfpages}
\usepackage{txfonts} % pxfonts txfonts 
\usepackage[T1]{fontenc}
\usepackage{lmodern}
\renewcommand*\familydefault{\sfdefault}
 \usepackage{euler}   % better than the option below
\usepackage{pdfsync}
\usepackage{multicol}
\newcommand{\ci}[1]{_{ {}_{\scriptstyle #1}}}
\graphicspath{ {images/} }


\newcommand{\norm}[1]{\ensuremath{\left\|#1\right\|}}
\newcommand{\abs}[1]{\ensuremath{\left\vert#1\right\vert}}
\newcommand{\ip}[2]{\ensuremath{\left\langle#1,#2\right\rangle}}
\newcommand{\p}{\ensuremath{\partial}}
\newcommand{\pr}{\mathcal{P}}

\newcommand{\pbar}{\ensuremath{\bar{\partial}}}
\newcommand{\db}{\overline\partial}
\newcommand{\D}{\mathbb{D}}
\newcommand{\B}{\mathbb{B}}
\newcommand{\Sp}{\mathbb{S}}
\newcommand{\T}{\mathbb{T}}
\newcommand{\R}{\mathbb{R}}
\newcommand{\Z}{\mathbb{Z}}
\newcommand{\C}{\mathbb{C}}
\newcommand{\N}{\mathbb{N}}
\newcommand{\Q}{\mathbb{Q}}
\newcommand{\mQ}{\mathcal{Q}}
\newcommand{\mS}{\mathcal{S}}
\newcommand{\scrH}{\mathcal{H}}
\newcommand{\scrL}{\mathcal{L}}
\newcommand{\td}{\widetilde\Delta}
\newcommand{\pw}{\text{PW}}
\newcommand{\esup}{\text{ess.sup}}
\newcommand{\Tn}{\mathcal{T}_n}
\newcommand{\Bn}{\mathbb{B}_n}
\newcommand{\rt}{\mathcal{O}}
\newcommand{\avg}[1]{\langle #1 \rangle}
\newcommand{\one}{\mathbbm{1}}
\newcommand{\eps}{\varepsilon}
\newcommand{\grad}{\nabla}

\newcommand{\La}{\langle }
\newcommand{\Ra}{\rangle }
\newcommand{\rk}{\operatorname{rk}}
\newcommand{\card}{\operatorname{card}}
\newcommand{\ran}{\operatorname{Ran}}
\newcommand{\osc}{\operatorname{OSC}}
\newcommand{\im}{\operatorname{Im}}
\newcommand{\re}{\operatorname{Re}}
\newcommand{\tr}{\operatorname{tr}}
\newcommand{\vf}{\varphi}
\newcommand{\f}[2]{\ensuremath{\frac{#1}{#2}}}

\newcommand{\kzp}{k_z^{(p,\alpha)}}
\newcommand{\klp}{k_{\lambda_i}^{(p,\alpha)}}
\newcommand{\TTp}{\mathcal{T}_p}
\newcommand{\m}[1]{\mathcal{#1}}
\newcommand{\md}{\mathcal{D}}
\newcommand{\qan}{\abs{Q}^{\alpha/n}}
\newcommand{\sbump}[2]{[[ #1,#2 ]]}
\newcommand{\mbump}[2]{\lceil #1,#2 \rceil}
\newcommand{\cbump}[2]{\lfloor #1,#2 \rfloor}

\newcommand{\hn}{{4}}
\newcommand{\dd}{{10-28}}
\newcommand{\class}{Aero 417}
\newcommand{\term}{Fall 2024}
\newcommand{\examnum}{Homework \hn: Due \dd}
\newcommand{\examdate}{}
\newcommand{\timelimit}{75 Minutes}
\newcommand{\vc}[3]{\langle #1,#2,#3\rangle}
\newcommand*{\vv}[1]{\vec{\mkern0mu#1}}
\newcommand{\bv}[1]{\boldsymbol{#1}}
\newcommand{\hide}[1]{}
\newcommand{\uvec}[1]{\boldsymbol{\hat{\textbf{#1}}}}
\newcommand{\vex}[1]{\boldsymbol{{\textbf{#1}}}}
\newcommand{\px}{\frac{\partial}{\partial x}}
\newcommand{\py}{\frac{\partial}{\partial y}}
\newcommand{\pt}{\frac{\partial}{\partial t}}
\newcommand{\pxx}{\frac{\partial^2}{\partial x^2}}
\newcommand{\pyy}{\frac{\partial^2}{\partial y^2}}
\newcommand{\ptt}{\frac{\partial^2}{\partial t^2}}


\pagestyle{head}
\firstpageheader{}{}{}
\runningheader{\class}{ Page \thepage\ of \numpages}{\examnum}
\runningheadrule

\makeatletter
\renewcommand*\env@matrix[1][*\c@MaxMatrixCols c]{%
  \hskip -\arraycolsep
  \let\@ifnextchar\new@ifnextchar
  \array{#1}}
\makeatother

\printanswers
\begin{document}

\noindent
\begin{tabular*}{\textwidth}{l @{\extracolsep{\fill}} r @{\extracolsep{6pt}} l}
\textbf{\class} & \textbf{Name:} & \makebox[2in]{\bf{Benjamin Tollison}}\\
\end{tabular*}\\
\rule[2ex]{\textwidth}{2pt}
\printanswers

\begin{document}

\noindent
\begin{tabular*}{\textwidth}{l @{\extracolsep{\fill}} r @{\extracolsep{6pt}} l}
\textbf{\class} & \textbf{Name:} & \makebox[2in]{\bf{Benjamin Tollison}}\\
\end{tabular*}\\
\rule[2ex]{\textwidth}{2pt}
%
\begin{questions}
\begin{question}
Comparing centrifugal and axial compressors, what is the reason in which centrifugal
compressor presents higher pressure ratio per stage than axial compressors?
\end{question}
\begin{solutionorbox}[\stretch{1}]
\end{solutionorbox}

\newpage 
\begin{question}
Based on a specific axial compressor design requirement, describe with details the procedure
to obtain an adequate preliminary 3D turbomachine sizing. Also, create a diagram of this
project process.
\end{question}
\begin{solutionorbox}[\stretch{1}]
\end{solutionorbox}


\newpage 
\begin{question}
What is the objective in the use of variable geometry (VIGV and VSV)?
\end{question}
\begin{solutionorbox}[\stretch{1}]
\end{solutionorbox}


\newpage 
\begin{question}
In a compressor operational map, for each rotational speed there are two limit points, stall
and choke. Make a sketch of a compressor map with different rotational speeds, identifying
these points including the stall/surge line and explain the physical aspects of stall, surge and
choke.
\end{question}
\begin{solutionorbox}[\stretch{1}]
\end{solutionorbox}


\end{questions}
\end{document}
