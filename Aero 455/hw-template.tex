\documentclass[12pt]{exam}
\usepackage[utf8]{inputenc}
\usepackage{amsmath,amstext,amsthm,amssymb,amsxtra, graphicx}
\usepackage[top=1.5in, bottom=1.5in, left=1.25in, right=1.25in]	{geometry}
%\usepackage[normalem]{ulem}
\usepackage{cancel}
\usepackage{txfonts} % pxfonts txfonts 
\usepackage[T1]{fontenc}
\usepackage{lmodern}
\renewcommand*\familydefault{\sfdefault}
 \usepackage{euler}   % better than the option below
\usepackage{pdfsync}
\usepackage{multicol}
\newcommand{\ci}[1]{_{ {}_{\scriptstyle #1}}}
\graphicspath{ {images/} }


\newcommand{\norm}[1]{\ensuremath{\left\|#1\right\|}}
\newcommand{\abs}[1]{\ensuremath{\left\vert#1\right\vert}}
\newcommand{\ip}[2]{\ensuremath{\left\langle#1,#2\right\rangle}}
\newcommand{\p}{\ensuremath{\partial}}
\newcommand{\pr}{\mathcal{P}}

\newcommand{\pbar}{\ensuremath{\bar{\partial}}}
\newcommand{\db}{\overline\partial}
\newcommand{\D}{\mathbb{D}}
\newcommand{\B}{\mathbb{B}}
\newcommand{\Sp}{\mathbb{S}}
\newcommand{\T}{\mathbb{T}}
\newcommand{\R}{\mathbb{R}}
\newcommand{\Z}{\mathbb{Z}}
\newcommand{\C}{\mathbb{C}}
\newcommand{\N}{\mathbb{N}}
\newcommand{\Q}{\mathbb{Q}}
\newcommand{\mQ}{\mathcal{Q}}
\newcommand{\mS}{\mathcal{S}}
\newcommand{\scrH}{\mathcal{H}}
\newcommand{\scrL}{\mathcal{L}}
\newcommand{\td}{\widetilde\Delta}
\newcommand{\pw}{\text{PW}}
\newcommand{\esup}{\text{ess.sup}}
\newcommand{\Tn}{\mathcal{T}_n}
\newcommand{\Bn}{\mathbb{B}_n}
\newcommand{\rt}{\mathcal{O}}
\newcommand{\avg}[1]{\langle #1 \rangle}
\newcommand{\one}{\mathbbm{1}}
\newcommand{\eps}{\varepsilon}
\newcommand{\grad}{\nabla}

\newcommand{\La}{\langle }
\newcommand{\Ra}{\rangle }
\newcommand{\rk}{\operatorname{rk}}
\newcommand{\card}{\operatorname{card}}
\newcommand{\ran}{\operatorname{Ran}}
\newcommand{\osc}{\operatorname{OSC}}
\newcommand{\im}{\operatorname{Im}}
\newcommand{\re}{\operatorname{Re}}
\newcommand{\tr}{\operatorname{tr}}
\newcommand{\vf}{\varphi}
\newcommand{\f}[2]{\ensuremath{\frac{#1}{#2}}}

\newcommand{\kzp}{k_z^{(p,\alpha)}}
\newcommand{\klp}{k_{\lambda_i}^{(p,\alpha)}}
\newcommand{\TTp}{\mathcal{T}_p}
\newcommand{\m}[1]{\mathcal{#1}}
\newcommand{\md}{\mathcal{D}}
\newcommand{\qan}{\abs{Q}^{\alpha/n}}
\newcommand{\sbump}[2]{[[ #1,#2 ]]}
\newcommand{\mbump}[2]{\lceil #1,#2 \rceil}
\newcommand{\cbump}[2]{\lfloor #1,#2 \rfloor}

\newcommand{\hn}{{1}}
\newcommand{\dd}{{09-27}}
\newcommand{\class}{Aero 455}
\newcommand{\term}{Fall 2024}
\newcommand{\examnum}{Homework \hn: Due \dd}
\newcommand{\examdate}{}
\newcommand{\timelimit}{75 Minutes}
\newcommand{\vc}[3]{\langle #1,#2,#3\rangle}
\newcommand*{\vv}[1]{\vec{\mkern0mu#1}}
\newcommand{\bv}[1]{\boldsymbol{#1}}
\newcommand{\hide}[1]{}
\newcommand{\uvec}[1]{\boldsymbol{\hat{\textbf{#1}}}}
\newcommand{\vex}[1]{\boldsymbol{{\textbf{#1}}}}
\newcommand{\px}{\frac{\partial}{\partial x}}
\newcommand{\py}{\frac{\partial}{\partial y}}
\newcommand{\pt}{\frac{\partial}{\partial t}}
\newcommand{\pxx}{\frac{\partial^2}{\partial x^2}}
\newcommand{\pyy}{\frac{\partial^2}{\partial y^2}}
\newcommand{\ptt}{\frac{\partial^2}{\partial t^2}}


\pagestyle{head}
\firstpageheader{}{}{}
\runningheader{\class}{ Page \thepage\ of \numpages}{\examnum}
\runningheadrule

\makeatletter
\renewcommand*\env@matrix[1][*\c@MaxMatrixCols c]{%
  \hskip -\arraycolsep
  \let\@ifnextchar\new@ifnextchar
  \array{#1}}
\makeatother

\printanswers
\begin{document}

\noindent
\begin{tabular*}{\textwidth}{l @{\extracolsep{\fill}} r @{\extracolsep{6pt}} l}
\textbf{\class} & \textbf{Name:} & \makebox[2in]{\bf{Benjamin Tollison}}\\
\end{tabular*}\\
\rule[2ex]{\textwidth}{2pt}
%
\begin{questions}
\begin{question}
Using Bernoulli’s equation instead of general energy equation, prove that the induced
velocity in the fully contracted wake of a rotor climbing vertically is twice the induced
velocity in the rotor plane. You could use the mass and momentum equations. 

\end{question}
\begin{solutionorbox}[\stretch{1}]
\begin{align*}
\begin{cases}
P_0 + \frac{1}{2}\rho v_0^2 = P_1 + \frac{1}{2}\rho v_1^2 \\
P_2 + \frac{1}{2}\rho v_2^2 = P_3 + \frac{1}{2}\rho v_3^2 \\ 
v_0 = 0 \\
v_1 = v_2 = v_i \\ 
v_3 = w
\end{cases}
\end{align*}
\begin{align*}
\begin{cases}
\rho v_i \sigma_i = \rho w \sigma_w \\
T = \int_{\sigma_3}{\rho v_3 \left(v_3 \cdot n\right)}d\sigma_3 - \int_{\sigma_0}{\rho \cancelto{0}{v_0} \left(\cancelto{0}{v_0}\cdot n\right)}d\sigma_0
\end{cases}
\end{align*}
Simplifying the thrust equation produces
\begin{equation}
T = \rho v_3^2 \sigma_3 = \dot{m}w = \rho w^2 \sigma_3
\end{equation}
Using the relation of \(P_0 = P_3\) and plugging back into Bernoulli’s produces

\[P_0 + \cancelto{0}{\frac{1}{2}\rho v_0^2} = P_1 + \frac{1}{2}\rho v_i^2 \]
\[P_0 = P_1 + \frac{1}{2}\rho v_i^2 \]
\[P_2 + \cancel{\frac{1}{2}\rho v_i^2} = P_1 + \cancel{\frac{1}{2}\rho v_i^2} + \frac{1}{2}\rho w^2 \]
\[P_2-P_0 = \frac{1}{2}\rho w^2\]
and the difference of pressure can be defined as the thrust across the blade membrane
\begin{equation}
\frac{T}{\sigma_2} = \frac{1}{2}\rho w^2
\end{equation}
Plugging equation 1 into 2 gives
\[\frac{\rho w^2 \sigma_3}{\sigma_2} = \frac{1}{2}\rho w^2\]
\[\frac{\cancel{\rho w^2} \sigma_3}{\sigma_2} = \frac{1}{2}\cancel{\rho w^2}\]
\begin{equation}
\frac{\sigma_3}{\sigma_2} = \frac{1}{2}
\end{equation}
Plugging equation 3 into the conservation of mass
\[\frac{\sigma_2}{\sigma_3} = \frac{w}{v_i}\]
\[2 = \frac{w}{v_i}\]
\[\therefore w = 2 v_i\]
\end{solutionorbox}

\newpage 
\begin{question}
Measurements have been made of rotor performance at a fixed rotor speed for a series of
blade pitch angles. The values of \(C_T\) that were measured were 6.0000E-06, 0.0010490,
0.0023760, 0.0040760 and 0.0055810, and the corresponding values of \(C_P\) were 0.000197,
0.000226, 0.000282, 0.000405 and 0.000555, respectively. Plot this data in the form of a
power polar (\(C_T\) vs. \(C_P\)). Explain (and show in a chart) how to extract induced power factor
(\(\kappa\) and zero thrust power (profile power) for the rotor from these measured data. Then, to
the experimental power polar chart add the analytical power polar curve predicted by
modified momentum theory
\end{question}
\begin{solutionorbox}[\stretch{1}]
\end{solutionorbox}


\newpage 
\begin{question}
Find all eigenvalues and eigenfunctions for the equation: 
$\varphi'' + \lambda \varphi = 0$ with $\varphi(0) = 0$ and $\varphi'(\pi) = 0.$
\end{question}
\begin{solutionorbox}[\stretch{1}]
\end{solutionorbox}


\newpage 
\begin{question}
Solve 
\begin{align*}
\begin{cases}
\ptt u = \pxx u\\ 
u(0, t) = u(\pi, t) = 0\\ 
u(x,0) = f(x), \pt u(x,0) = 0,
\end{cases}
\end{align*}
where $f(x)$ is the ``hat function'': it's $0$ and $0$ and $\pi$ and $1$ and $\pi/2$ and 
linear in between. 
\end{question}
\begin{solutionorbox}[\stretch{1}]
\end{solutionorbox}


\newpage 
\begin{question}
This is another perspective on the SOV method. Consider the problem 
\begin{align*}
\begin{cases}
\pt u = \pxx u\\
u(0,t) = u(\pi, t) = 0\\ 
u(x,0) = g(x)
\end{cases}.
\end{align*}
For each fixed $x$, assume that the solution can be written as $\sum_{k=1}^{\infty}B_k\sin(kx)$. 
Note that the $B_k$ depend on $t$ so a better way to write it is $\sum_{k=1}^{\infty}
B_k(t) \sin(kx)$. Starting from this point, find the SOV solution. 
(Note: I had a typo that said $B_k(x)$ before; it should be 
$B_k(t)$ SORRY!!)
\end{question}
\begin{solutionorbox}[\stretch{1}]
Plugging the function into the PDE we get: 
\begin{align*}
\sum_{k=1}^{\infty}B_k'(t)\sin kx
&= \sum_{k=1}^{\infty}B_k(t)\frac{d^2}{dx^2}\sin kx
\\&= \sum_{k=1}^{\infty}B_k(t)(-k^2 \sin kx)
\\&= -\sum_{k=1}^{\infty}B_k(t)k^2 \sin kx.
\end{align*}
Subtracting the left from the right: 
\begin{align*}
0 = \sum_{k=1}^{\infty}(B_k'(t) + k^2 B_k(t))\sin kx.
\end{align*}
Since the $\sin kx$ functions are linearly independent, this implies that
for all $k$, $B_k'(t) =- k^2 B_k(t)$. So we get an ODE for $B_k$ and the 
solution is $B_k (t) = B_k(0)e^{-k^2 t}$. To find the values of the 
constants note that: 
\begin{align*}
g(x) 
= u(x,0)
= \sum_{k=1}^{\infty}B_k(0) \sin kx.
\end{align*}
That is, as before, $B_k(0) = \frac{2}{\pi}\int_{0}^{\pi}g(y)\sin ky dy$.
\end{solutionorbox}

\end{questions}
\end{document}
